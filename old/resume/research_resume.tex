\documentclass[margin,line]{res}
%\usepackage{layout}

%\setlength{\voffset}{0in}
%\setlength{\headsep}{5pt}
\usepackage{amssymb}
\usepackage[bookmarks,bookmarksopen,bookmarksdepth=2]{hyperref}
\usepackage{xcolor}
\hypersetup{
    colorlinks,
    linkcolor={red!50!black},
    citecolor={blue!50!black},
    urlcolor={blue!80!black}
}
\pagestyle{empty}
\topmargin -0.8in
\oddsidemargin -.7in
\evensidemargin -.6in
\textwidth=6.5in
\textheight 10.5in
\itemsep=0in
\parsep=0in
% if using pdflatex:
\setlength{\pdfpagewidth}{\paperwidth}
\setlength{\pdfpageheight}{\paperheight} 

\newenvironment{list1}{
  \begin{list}{\ding{113}}{%
      \setlength{\itemsep}{0in}
      \setlength{\parsep}{0in} \setlength{\parskip}{0in}
      \setlength{\topsep}{0in} \setlength{\partopsep}{0in} 
      \setlength{\leftmargin}{0.17in}}}{\end{list}}
\newenvironment{list2}{
  \begin{list}{$\bullet$}{%
      \setlength{\itemsep}{0in}
      \setlength{\parsep}{0in} \setlength{\parskip}{0in}
      \setlength{\topsep}{0in} \setlength{\partopsep}{0in} 
      \setlength{\leftmargin}{0.2in}}}{\end{list}}


\begin{document}

\name{Fahim Tahmid Chowdhury \vspace*{.1in}}

\begin{tabular}{@{}p{2.5in}p{4in}}
\hspace{1cm} & {\it Mobile:}  \href{tel:+17864062617}{+1(786)-406/2617} \\
\hspace{1cm} & {\it Skype:}  \href{skype:fahim.tahmid.chowdhury?call}{fahim.tahmid.chowdhury} \\
\hspace{1cm} & {\it E-mail:} \href{mailto:fchowdhu@cs.fsu.edu}{\nolinkurl{fchowdhu@cs.fsu.edu} } \\
\hspace{1cm} & {\it Website:} \href{http://ww2.cs.fsu.edu/~fchowdhu}{http://ww2.cs.fsu.edu/\textasciitilde fchowdhu} \\
\hspace{1cm} & {\it LinkedIn:} \href{http://www.linkedin.com/in/fahimtahmidchowdhury}{http://www.linkedin.com/in/fahimtahmidchowdhury} \\
\end{tabular}

\vspace*{-.2in}

\begin{resume}

\section{\sc Education}
{\bf \href{http://www.fsu.edu}{Florida State University}}, Tallahassee, Florida\\
\vspace*{-.15in}
\begin{list1}
\item[] Ph.D. candidate, Computer Science, currently enrolled, \textit{CGPA} \textbf{4.00}/4.00 (Intended grad: \bf{January 2022})
\end{list1}
\vspace*{-.15in}
{\bf \href{http://buet.ac.bd}{Bangladesh University of Engineering and Technology}}, Dhaka, Bangladesh\\
\vspace*{-.15in}
\begin{list1}
\item[] BSc., Computer Science and Engineering, February 2013, \textit{CGPA} \textbf{3.54}/4.00
\end{list1}

\vspace*{-.1in}

\section{\sc Research Interests}
\begin{list2}
\item \textbf{High Performance Computing (HPC) Systems:} Domain-specific Systems Design, HPC I/O Optimization, Heterogeneous Storage Stack, HPC File Systems, HPC Workflow, Performance Analysis
\item \textbf{Artificial Intelligence:} Deep Learning(DL) at Scale, Reinforcement Learning
\end{list2}

\vspace*{-.1in}

\section{\sc Research Experience}
{\bf \href{http://www.cs.fsu.edu/}{Department of Computer Science, Florida State University}}

\vspace{-.4cm}
\textbf{{\em Graduate Research Assistant}} \hfill {\bf August 2017 - Present}\\
- Ph.D. student researcher at \textit{\href{http://castl.cs.fsu.edu/doku.php/}{Computer Architecture and SysTems Research Lab (CASTL)}} supervised by \textit{\href{https://www.cs.fsu.edu/~yuw/}{Professor Dr. Weikuan Yu}}, specializing in domain-specific distributed systems design

\vspace{-.3cm}
{\bf \href{https://computing.llnl.gov/casc}{Center for Applied Scientific Computing (CASC)}}, \href{http://www.llnl.gov/}{Lawrence Livermore National Laboratory (LLNL)}

\vspace{-.4cm}
\textbf{{\em Student Intern}} \hfill {\bf May 2019 - August 2019}\\
- Worked on optimizing I/O strategies in HPC application workflows like Cancer Moonshot Pilot 2 in the \href{https://computing.llnl.gov/casc/data-analysis-group}{Data Analysis Group} at CASC. Achieved 84.7\% latency improvement by using burst buffers on Lassen.

\vspace{-.3cm}
{\bf \href{http://www.nersc.gov/}{National Energy Research Scientific Computing Center (NERSC)}}, \href{http://www.lbl.gov/}{Lawrence Berkeley National Laboratory (LBNL)}, Berkeley, California

\vspace{-.4cm}
\textbf{{\em Student Assistant}}(Summer intern) \hfill {\bf May 2018 - August 2018}\\
- Worked in the \href{http://www.nersc.gov/users/data-analytics/}{Data Analytics and Services} group at NERSC. Analyzed scalable data pipeline for distributed DL atop TensorFlow and Horovod. Determined I/O bottleneck of upto 11.04\% in DL training time.

\vspace{-.3cm}
{\bf \href{http://www.nersc.gov/}{NERSC}}, \href{http://www.lbl.gov/}{LBNL}

\vspace{-.4cm}
\textbf{{\em LBNL Affiliate}} \hfill {\bf August 2018 - August 2019}\\
- Enhanced the summer internship project on determining I/O bottlenecks in distributed DL applications.

\vspace*{-.1in}

\section{\sc Research Projects}
\begin{list2}
\item {\bf HPC Workflow I/O Optimization:} Built an Emulator during internship at CASC to analyze different HPC I/O patterns, e.g., Deep Learning Training I/O, Checkpoint/Restart, Producer-Consumer, etc. Developing a middleware to implement optimization strategies for application workflows.
\item {\bf BeeGFS Performance Evaluation:} Published a research paper on the performance evaluation of \href{https://www.beegfs.io/content/}{BeeGFS} parallel cluster file system using IOR and MDTest, and deep learning applications atop TensorFlow, Horovod and LBANN. Currently, working on analyzing the fitness of \href{http://www.beegfs.io/wiki/BeeOND}{BeeGFS On Demand (BeeOND)} as an ephemeral burst buffer file system.
\item {\bf Scalable Data Pipeline for Distributed Deep Learning:} Analyzed and profiled I/O behavior posed by cutting-edge deep learning applications at scale by using a \href{https://github.com/NERSC/DL-Parallel-IO}{logging framework} developed during internship at NERSC. Pinpointed I/O bottlenecks caused by metadata overhead in deep learning training.
\end{list2}

\vspace*{-.1in}

\section{\sc Publications}
\begin{list2}
\item[ - ] {\bf F. Chowdhury}, F. Di Natale, A. Moody, E. Gonsiorowski, K. Mohror, and W. Yu. ``Understanding I/O Behavior in Scientific Workflows on High Performance Computing Systems," in Proceedings of the \textit{International Conference on High Performance Computing, Networking, Storage and Analysis 2019 (SC19)}, \href{http://sc19.supercomputing.org/proceedings/tech_poster/tech_poster_pages/rpost258.html}{Regular Poster}, Nov. 2019.
\item[ - ] {\bf F. Chowdhury}, Y. Zhu, T. Heer, S. Paredes, A. Moody, R. Goldstone, K. Mohror, and W. Yu, ``I/O Characterization and Performance Evaluation of BeeGFS for Deep Learning," in Proceedings of the \textit{48th International Conference on Parallel Processing (ICPP 2019)}, \href{http://www.osti.gov/servlets/purl/1559405}{Research Paper}, Aug. 2019.
\item[ - ] {\bf F. Chowdhury}, J. Liu, Q. Koziol, T. Kurth, S. Farrell, S. Byna, Prabhat, and W. Yu, ``Initial Characterization of I/O in Large-Scale Deep Learning Applications," in \textit{SC18, 3RD Joint International Workshop on Parallel Data Storage \& Data Intensive Scalable Computing Systems (PDSW-DISCS 2018)}, \href{http://www.pdsw.org/pdsw-discs18/wips/abstracts/chowdhury-wip-pdsw-discs18.pdf}{Work-in-progress (WIP) Abstract}, Nov. 2018.
\item[ - ] Y. Zhu, {\bf F. Chowdhury}, H. Fu, A. Moody, K. Mohror, K. Sato, and W. Yu, ``Entropy-Aware I/O Pipelining for Large-Scale Deep Learning on HPC Systems," in \textit{IEEE International Symposium on the Modeling, Analysis, and Simulation of Computer and Telecommunication Systems (MASCOTS 2018)}, \href{http://www.mscs.mu.edu/~mascots/Papers/72.pdf}{Research Paper}, Sep. 2018.
\end{list2}

\vspace*{-.1in}

\section{\sc Technical \\Skills}
\begin{list2}
\item[ - ] Programming Languages: {\bf C/C\texttt{++}}, {\bf Python}, C\texttt{\#}, MATLAB, Java, Javascript
\item[ - ] Libraries: {\bf MPI}, {\bf HDF5}, BSD sockets, WinSock, OpenGL, Boost, Windows API, Google Test
\item[ - ] Frameworks: {\bf TensorFlow}, {\bf Horovod}, {\bf LBANN}, Qt Framework, MFC, .NET Framework
\item[ - ] Distributed File Systems: {\bf BeeGFS}, Lustre, BurstFS, UnifyCR
\end{list2}

\end{resume}
\end{document}
